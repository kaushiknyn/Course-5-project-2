% Options for packages loaded elsewhere
\PassOptionsToPackage{unicode}{hyperref}
\PassOptionsToPackage{hyphens}{url}
%
\documentclass[
]{article}
\usepackage{lmodern}
\usepackage{amssymb,amsmath}
\usepackage{ifxetex,ifluatex}
\ifnum 0\ifxetex 1\fi\ifluatex 1\fi=0 % if pdftex
  \usepackage[T1]{fontenc}
  \usepackage[utf8]{inputenc}
  \usepackage{textcomp} % provide euro and other symbols
\else % if luatex or xetex
  \usepackage{unicode-math}
  \defaultfontfeatures{Scale=MatchLowercase}
  \defaultfontfeatures[\rmfamily]{Ligatures=TeX,Scale=1}
\fi
% Use upquote if available, for straight quotes in verbatim environments
\IfFileExists{upquote.sty}{\usepackage{upquote}}{}
\IfFileExists{microtype.sty}{% use microtype if available
  \usepackage[]{microtype}
  \UseMicrotypeSet[protrusion]{basicmath} % disable protrusion for tt fonts
}{}
\makeatletter
\@ifundefined{KOMAClassName}{% if non-KOMA class
  \IfFileExists{parskip.sty}{%
    \usepackage{parskip}
  }{% else
    \setlength{\parindent}{0pt}
    \setlength{\parskip}{6pt plus 2pt minus 1pt}}
}{% if KOMA class
  \KOMAoptions{parskip=half}}
\makeatother
\usepackage{xcolor}
\IfFileExists{xurl.sty}{\usepackage{xurl}}{} % add URL line breaks if available
\IfFileExists{bookmark.sty}{\usepackage{bookmark}}{\usepackage{hyperref}}
\hypersetup{
  pdftitle={Analysis of Severe Weather Events on Public Health and Economy for Communities and Municipalities in the United States},
  pdfauthor={Kaushik Sivasankaran},
  hidelinks,
  pdfcreator={LaTeX via pandoc}}
\urlstyle{same} % disable monospaced font for URLs
\usepackage[margin=1in]{geometry}
\usepackage{color}
\usepackage{fancyvrb}
\newcommand{\VerbBar}{|}
\newcommand{\VERB}{\Verb[commandchars=\\\{\}]}
\DefineVerbatimEnvironment{Highlighting}{Verbatim}{commandchars=\\\{\}}
% Add ',fontsize=\small' for more characters per line
\usepackage{framed}
\definecolor{shadecolor}{RGB}{248,248,248}
\newenvironment{Shaded}{\begin{snugshade}}{\end{snugshade}}
\newcommand{\AlertTok}[1]{\textcolor[rgb]{0.94,0.16,0.16}{#1}}
\newcommand{\AnnotationTok}[1]{\textcolor[rgb]{0.56,0.35,0.01}{\textbf{\textit{#1}}}}
\newcommand{\AttributeTok}[1]{\textcolor[rgb]{0.77,0.63,0.00}{#1}}
\newcommand{\BaseNTok}[1]{\textcolor[rgb]{0.00,0.00,0.81}{#1}}
\newcommand{\BuiltInTok}[1]{#1}
\newcommand{\CharTok}[1]{\textcolor[rgb]{0.31,0.60,0.02}{#1}}
\newcommand{\CommentTok}[1]{\textcolor[rgb]{0.56,0.35,0.01}{\textit{#1}}}
\newcommand{\CommentVarTok}[1]{\textcolor[rgb]{0.56,0.35,0.01}{\textbf{\textit{#1}}}}
\newcommand{\ConstantTok}[1]{\textcolor[rgb]{0.00,0.00,0.00}{#1}}
\newcommand{\ControlFlowTok}[1]{\textcolor[rgb]{0.13,0.29,0.53}{\textbf{#1}}}
\newcommand{\DataTypeTok}[1]{\textcolor[rgb]{0.13,0.29,0.53}{#1}}
\newcommand{\DecValTok}[1]{\textcolor[rgb]{0.00,0.00,0.81}{#1}}
\newcommand{\DocumentationTok}[1]{\textcolor[rgb]{0.56,0.35,0.01}{\textbf{\textit{#1}}}}
\newcommand{\ErrorTok}[1]{\textcolor[rgb]{0.64,0.00,0.00}{\textbf{#1}}}
\newcommand{\ExtensionTok}[1]{#1}
\newcommand{\FloatTok}[1]{\textcolor[rgb]{0.00,0.00,0.81}{#1}}
\newcommand{\FunctionTok}[1]{\textcolor[rgb]{0.00,0.00,0.00}{#1}}
\newcommand{\ImportTok}[1]{#1}
\newcommand{\InformationTok}[1]{\textcolor[rgb]{0.56,0.35,0.01}{\textbf{\textit{#1}}}}
\newcommand{\KeywordTok}[1]{\textcolor[rgb]{0.13,0.29,0.53}{\textbf{#1}}}
\newcommand{\NormalTok}[1]{#1}
\newcommand{\OperatorTok}[1]{\textcolor[rgb]{0.81,0.36,0.00}{\textbf{#1}}}
\newcommand{\OtherTok}[1]{\textcolor[rgb]{0.56,0.35,0.01}{#1}}
\newcommand{\PreprocessorTok}[1]{\textcolor[rgb]{0.56,0.35,0.01}{\textit{#1}}}
\newcommand{\RegionMarkerTok}[1]{#1}
\newcommand{\SpecialCharTok}[1]{\textcolor[rgb]{0.00,0.00,0.00}{#1}}
\newcommand{\SpecialStringTok}[1]{\textcolor[rgb]{0.31,0.60,0.02}{#1}}
\newcommand{\StringTok}[1]{\textcolor[rgb]{0.31,0.60,0.02}{#1}}
\newcommand{\VariableTok}[1]{\textcolor[rgb]{0.00,0.00,0.00}{#1}}
\newcommand{\VerbatimStringTok}[1]{\textcolor[rgb]{0.31,0.60,0.02}{#1}}
\newcommand{\WarningTok}[1]{\textcolor[rgb]{0.56,0.35,0.01}{\textbf{\textit{#1}}}}
\usepackage{graphicx,grffile}
\makeatletter
\def\maxwidth{\ifdim\Gin@nat@width>\linewidth\linewidth\else\Gin@nat@width\fi}
\def\maxheight{\ifdim\Gin@nat@height>\textheight\textheight\else\Gin@nat@height\fi}
\makeatother
% Scale images if necessary, so that they will not overflow the page
% margins by default, and it is still possible to overwrite the defaults
% using explicit options in \includegraphics[width, height, ...]{}
\setkeys{Gin}{width=\maxwidth,height=\maxheight,keepaspectratio}
% Set default figure placement to htbp
\makeatletter
\def\fps@figure{htbp}
\makeatother
\setlength{\emergencystretch}{3em} % prevent overfull lines
\providecommand{\tightlist}{%
  \setlength{\itemsep}{0pt}\setlength{\parskip}{0pt}}
\setcounter{secnumdepth}{-\maxdimen} % remove section numbering

\title{Analysis of Severe Weather Events on Public Health and Economy for
Communities and Municipalities in the United States}
\author{Kaushik Sivasankaran}
\date{8/16/2020}

\begin{document}
\maketitle

\hypertarget{synopsis}{%
\subsection{Synopsis}\label{synopsis}}

Storms and other severe events can cause both public health and economic
problems for communities and municipalities in the United States of
America. This analysis involves exploring the U.S. National Oceanic and
Atmospheric Administration's (NOAA) storm database to determine which
types of severe events are most harmful to population health and have
the greatest adverse econmic impact in the United States. This analysis
will use a database that comprises details of severe weather events from
1950 ranging till November 2011.

\hypertarget{data-processing}{%
\subsection{Data Processing}\label{data-processing}}

First, the data required to perform the analysis needs to be extracted.

After downloading the data from the course project website into the R
project directory, the next step is to read in the \emph{csv.bz2} zip
file.

\begin{Shaded}
\begin{Highlighting}[]
\NormalTok{storm_data <-}\StringTok{ }\KeywordTok{read.csv}\NormalTok{(}\DataTypeTok{file =} \StringTok{"repdata_data_StormData.csv.bz2"}\NormalTok{, }\DataTypeTok{header =} \OtherTok{TRUE}\NormalTok{, }\DataTypeTok{sep =} \StringTok{","}\NormalTok{, }\DataTypeTok{as.is =} \OtherTok{TRUE}\NormalTok{)}
\end{Highlighting}
\end{Shaded}

\hypertarget{extracting-the-required-data}{%
\subsubsection{Extracting the required
data}\label{extracting-the-required-data}}

In order to calculate the event types that are most harmful with respect
to population health and economic impact, we would need to extract
following data:

\begin{itemize}
\tightlist
\item
  EVETYPE - Event type (Ex: Tornado, Blizzard, etc.)
\item
  Health related impacts:

  \begin{enumerate}
  \def\labelenumi{\arabic{enumi}.}
  \tightlist
  \item
    Fatalities - Number of deaths caused by severe events
  \item
    Injuries - Number of injuries caused by severe events
  \end{enumerate}
\item
  Economic related impacts:

  \begin{enumerate}
  \def\labelenumi{\arabic{enumi}.}
  \tightlist
  \item
    PROPDMG - Property related damage
  \item
    PROPDMGEXP - The unit for property damage by value
  \item
    CROPDMG - Crop related damage
  \item
    CROPDMGEXP - The unit for crop damage by value
  \end{enumerate}
\end{itemize}

\begin{Shaded}
\begin{Highlighting}[]
\CommentTok{#Extracting only relevant data from storms data}
\NormalTok{event_data <-}\StringTok{ }\NormalTok{storm_data[, }\KeywordTok{c}\NormalTok{(}\StringTok{"EVTYPE"}\NormalTok{, }\StringTok{"FATALITIES"}\NormalTok{, }\StringTok{"INJURIES"}\NormalTok{, }\StringTok{"PROPDMG"}\NormalTok{, }\StringTok{"PROPDMGEXP"}\NormalTok{, }\StringTok{"CROPDMG"}\NormalTok{, }\StringTok{"CROPDMGEXP"}\NormalTok{)]}

\CommentTok{# Checking for NA values in the new dataframe}

\KeywordTok{sum}\NormalTok{(}\KeywordTok{is.na}\NormalTok{(event_data))}
\end{Highlighting}
\end{Shaded}

\begin{verbatim}
## [1] 0
\end{verbatim}

\begin{Shaded}
\begin{Highlighting}[]
\CommentTok{#Sorting the data by event type to identify the top 10 event types}

\KeywordTok{sort}\NormalTok{(}\KeywordTok{table}\NormalTok{(event_data}\OperatorTok{$}\NormalTok{EVTYPE),}\DataTypeTok{decreasing =} \OtherTok{TRUE}\NormalTok{)[}\DecValTok{1}\OperatorTok{:}\DecValTok{10}\NormalTok{]}
\end{Highlighting}
\end{Shaded}

\begin{verbatim}
## 
##               HAIL          TSTM WIND  THUNDERSTORM WIND            TORNADO 
##             288661             219940              82563              60652 
##        FLASH FLOOD              FLOOD THUNDERSTORM WINDS          HIGH WIND 
##              54277              25326              20843              20212 
##          LIGHTNING         HEAVY SNOW 
##              15754              15708
\end{verbatim}

Now in order to identify the event type better, we will group event
types with common words as one event type. For example, TSTM WIND,
THUNDERSTORM WIND, etc., will be classified as event type WIND and
unclassifiable event types as OTHER

\begin{Shaded}
\begin{Highlighting}[]
\CommentTok{# Creating a new variable to store the combined event type}
\NormalTok{event_data}\OperatorTok{$}\NormalTok{Event <-}\StringTok{ "OTHER"}

\CommentTok{#Group by keyword in event type}
\NormalTok{event_data}\OperatorTok{$}\NormalTok{Event[}\KeywordTok{grep}\NormalTok{(}\DataTypeTok{pattern =} \StringTok{"HAIL"}\NormalTok{, }\DataTypeTok{x =}\NormalTok{ event_data}\OperatorTok{$}\NormalTok{EVTYPE, }\DataTypeTok{ignore.case =} \OtherTok{TRUE}\NormalTok{)] <-}\StringTok{ "HAIL"}
\NormalTok{event_data}\OperatorTok{$}\NormalTok{Event[}\KeywordTok{grep}\NormalTok{(}\DataTypeTok{pattern =} \StringTok{"WIND"}\NormalTok{, }\DataTypeTok{x =}\NormalTok{ event_data}\OperatorTok{$}\NormalTok{EVTYPE, }\DataTypeTok{ignore.case =} \OtherTok{TRUE}\NormalTok{)] <-}\StringTok{ "WIND"}
\NormalTok{event_data}\OperatorTok{$}\NormalTok{Event[}\KeywordTok{grep}\NormalTok{(}\DataTypeTok{pattern =} \StringTok{"SNOW"}\NormalTok{, }\DataTypeTok{x =}\NormalTok{ event_data}\OperatorTok{$}\NormalTok{EVTYPE, }\DataTypeTok{ignore.case =} \OtherTok{TRUE}\NormalTok{)] <-}\StringTok{ "SNOW"}
\NormalTok{event_data}\OperatorTok{$}\NormalTok{Event[}\KeywordTok{grep}\NormalTok{(}\DataTypeTok{pattern =} \StringTok{"FLOOD"}\NormalTok{, }\DataTypeTok{x =}\NormalTok{ event_data}\OperatorTok{$}\NormalTok{EVTYPE, }\DataTypeTok{ignore.case =} \OtherTok{TRUE}\NormalTok{)] <-}\StringTok{ "FLOOD"}
\NormalTok{event_data}\OperatorTok{$}\NormalTok{Event[}\KeywordTok{grep}\NormalTok{(}\DataTypeTok{pattern =} \StringTok{"HAIL"}\NormalTok{, }\DataTypeTok{x =}\NormalTok{ event_data}\OperatorTok{$}\NormalTok{EVTYPE, }\DataTypeTok{ignore.case =} \OtherTok{TRUE}\NormalTok{)] <-}\StringTok{ "HAIL"}
\NormalTok{event_data}\OperatorTok{$}\NormalTok{Event[}\KeywordTok{grep}\NormalTok{(}\DataTypeTok{pattern =} \StringTok{"RAIN"}\NormalTok{, }\DataTypeTok{x =}\NormalTok{ event_data}\OperatorTok{$}\NormalTok{EVTYPE, }\DataTypeTok{ignore.case =} \OtherTok{TRUE}\NormalTok{)] <-}\StringTok{ "RAIN"}
\NormalTok{event_data}\OperatorTok{$}\NormalTok{Event[}\KeywordTok{grep}\NormalTok{(}\DataTypeTok{pattern =} \StringTok{"TORNADO"}\NormalTok{, }\DataTypeTok{x =}\NormalTok{ event_data}\OperatorTok{$}\NormalTok{EVTYPE, }\DataTypeTok{ignore.case =} \OtherTok{TRUE}\NormalTok{)] <-}\StringTok{ "TORNADO"}
\NormalTok{event_data}\OperatorTok{$}\NormalTok{Event[}\KeywordTok{grep}\NormalTok{(}\DataTypeTok{pattern =} \StringTok{"LIGHTNING"}\NormalTok{, }\DataTypeTok{x =}\NormalTok{ event_data}\OperatorTok{$}\NormalTok{EVTYPE, }\DataTypeTok{ignore.case =} \OtherTok{TRUE}\NormalTok{)] <-}\StringTok{ "LIGHTNING"}
\NormalTok{event_data}\OperatorTok{$}\NormalTok{Event[}\KeywordTok{grep}\NormalTok{(}\DataTypeTok{pattern =} \StringTok{"RAIN"}\NormalTok{, }\DataTypeTok{x =}\NormalTok{ event_data}\OperatorTok{$}\NormalTok{EVTYPE, }\DataTypeTok{ignore.case =} \OtherTok{TRUE}\NormalTok{)] <-}\StringTok{ "RAIN"}
\NormalTok{event_data}\OperatorTok{$}\NormalTok{Event[}\KeywordTok{grep}\NormalTok{(}\DataTypeTok{pattern =} \StringTok{"HEAT"}\NormalTok{, }\DataTypeTok{x =}\NormalTok{ event_data}\OperatorTok{$}\NormalTok{EVTYPE, }\DataTypeTok{ignore.case =} \OtherTok{TRUE}\NormalTok{)] <-}\StringTok{ "HEAT"}
\NormalTok{event_data}\OperatorTok{$}\NormalTok{Event[}\KeywordTok{grep}\NormalTok{(}\DataTypeTok{pattern =} \StringTok{"STORM"}\NormalTok{, }\DataTypeTok{x =}\NormalTok{ event_data}\OperatorTok{$}\NormalTok{EVTYPE, }\DataTypeTok{ignore.case =} \OtherTok{TRUE}\NormalTok{)] <-}\StringTok{ "STORM"}

\CommentTok{# Sorting on the new grouped event types}

\KeywordTok{sort}\NormalTok{(}\KeywordTok{table}\NormalTok{(event_data}\OperatorTok{$}\NormalTok{Event), }\DataTypeTok{decreasing =} \OtherTok{TRUE}\NormalTok{)}
\end{Highlighting}
\end{Shaded}

\begin{verbatim}
## 
##      HAIL      WIND     STORM     FLOOD   TORNADO     OTHER      SNOW LIGHTNING 
##    290306    254321    124624     82695     60700     41370     17633     15762 
##      RAIN      HEAT 
##     12238      2648
\end{verbatim}

As seen by the sorting above, it is now clear that most of the severe
weather events were contributed by WIND and HAIL type events

Next, lets analyze the dollar unit values for any inconsistencies

\begin{Shaded}
\begin{Highlighting}[]
\KeywordTok{sort}\NormalTok{(}\KeywordTok{table}\NormalTok{(event_data}\OperatorTok{$}\NormalTok{PROPDMGEXP), }\DataTypeTok{decreasing =} \OtherTok{TRUE}\NormalTok{)[}\DecValTok{1}\OperatorTok{:}\DecValTok{10}\NormalTok{]}
\end{Highlighting}
\end{Shaded}

\begin{verbatim}
## 
##             K      M      0      B      5      1      2      ?      m 
## 465934 424665  11330    216     40     28     25     13      8      7
\end{verbatim}

\begin{Shaded}
\begin{Highlighting}[]
\KeywordTok{sort}\NormalTok{(}\KeywordTok{table}\NormalTok{(event_data}\OperatorTok{$}\NormalTok{CROPDMGEXP), }\DataTypeTok{decreasing =} \OtherTok{TRUE}\NormalTok{)[}\DecValTok{1}\OperatorTok{:}\DecValTok{10}\NormalTok{]}
\end{Highlighting}
\end{Shaded}

\begin{verbatim}
## 
##             K      M      k      0      B      ?      2      m   <NA> 
## 618413 281832   1994     21     19      9      7      1      1
\end{verbatim}

There seems to be some inconsistencies in the units. So we can transform
those variables in one unit (dollar) variable using the following rules:
* K or k: thousand dollars (10\^{}3) * M or m: million dollars (10\^{}6)
* B or b: billion dollars (10\^{}9) * The rest would be considered as
dollars

The new variables are a product of damage value and dollar unit

\begin{Shaded}
\begin{Highlighting}[]
\NormalTok{event_data}\OperatorTok{$}\NormalTok{PROPDMGEXP[}\KeywordTok{is.na}\NormalTok{(event_data}\OperatorTok{$}\NormalTok{PROPDMGEXP)] <-}\StringTok{ "0"} \CommentTok{#NA's considered as unit dollars}

\NormalTok{event_data}\OperatorTok{$}\NormalTok{PROPDMGEXP[}\OperatorTok{!}\KeywordTok{grep}\NormalTok{(}\StringTok{"K|M|B"}\NormalTok{,}\DataTypeTok{x =}\NormalTok{ event_data}\OperatorTok{$}\NormalTok{PROPDMGEXP, }\DataTypeTok{ignore.case =} \OtherTok{TRUE}\NormalTok{)] <-}\StringTok{ "0"} \CommentTok{#Anything other than K,M,B are considered unit dollars}

\NormalTok{event_data}\OperatorTok{$}\NormalTok{PROPDMGEXP[}\KeywordTok{grep}\NormalTok{(}\StringTok{"K"}\NormalTok{, }\DataTypeTok{x =}\NormalTok{ event_data}\OperatorTok{$}\NormalTok{PROPDMGEXP, }\DataTypeTok{ignore.case =} \OtherTok{TRUE}\NormalTok{)] <-}\StringTok{ "3"}

\NormalTok{event_data}\OperatorTok{$}\NormalTok{PROPDMGEXP[}\KeywordTok{grep}\NormalTok{(}\StringTok{"M"}\NormalTok{, }\DataTypeTok{x =}\NormalTok{ event_data}\OperatorTok{$}\NormalTok{PROPDMGEXP,}\DataTypeTok{ignore.case =} \OtherTok{TRUE}\NormalTok{)] <-}\StringTok{ "6"}

\NormalTok{event_data}\OperatorTok{$}\NormalTok{PROPDMGEXP[}\KeywordTok{grep}\NormalTok{(}\StringTok{"B"}\NormalTok{, }\DataTypeTok{x =}\NormalTok{ event_data}\OperatorTok{$}\NormalTok{PROPDMGEXP,}\DataTypeTok{ignore.case =} \OtherTok{TRUE}\NormalTok{)] <-}\StringTok{ "9"}

\CommentTok{#Convert unit field to numeric}
\NormalTok{event_data}\OperatorTok{$}\NormalTok{PROPDMGEXP <-}\StringTok{ }\KeywordTok{as.numeric}\NormalTok{(event_data}\OperatorTok{$}\NormalTok{PROPDMGEXP)}
\end{Highlighting}
\end{Shaded}

\begin{verbatim}
## Warning: NAs introduced by coercion
\end{verbatim}

\begin{Shaded}
\begin{Highlighting}[]
\CommentTok{#Creating a new variable Property damage that would be the product of the dollar value and the unit}
\NormalTok{event_data}\OperatorTok{$}\NormalTok{Property_Damage <-}\StringTok{ }\NormalTok{event_data}\OperatorTok{$}\NormalTok{PROPDMG }\OperatorTok{*}\StringTok{ }\DecValTok{10}\OperatorTok{^}\NormalTok{event_data}\OperatorTok{$}\NormalTok{PROPDMGEXP}

\CommentTok{#Doing the same with crop damage}
\NormalTok{event_data}\OperatorTok{$}\NormalTok{CROPDMGEXP[}\KeywordTok{is.na}\NormalTok{(event_data}\OperatorTok{$}\NormalTok{CROPDMGEXP)] <-}\StringTok{ "0"} \CommentTok{#NA's considered as unit dollars}

\NormalTok{event_data}\OperatorTok{$}\NormalTok{CROPDMGEXP[}\OperatorTok{!}\KeywordTok{grep}\NormalTok{(}\StringTok{"K|M|B"}\NormalTok{, }\DataTypeTok{x =}\NormalTok{ event_data}\OperatorTok{$}\NormalTok{CROPDMGEXP, }\DataTypeTok{ignore.case =} \OtherTok{TRUE}\NormalTok{)] <-}\StringTok{ "0"} \CommentTok{#Anything other than K,M,B are considered unit dollars}

\NormalTok{event_data}\OperatorTok{$}\NormalTok{CROPDMGEXP[}\KeywordTok{grep}\NormalTok{(}\StringTok{"K"}\NormalTok{, }\DataTypeTok{x =}\NormalTok{ event_data}\OperatorTok{$}\NormalTok{CROPDMGEXP, }\DataTypeTok{ignore.case =} \OtherTok{TRUE}\NormalTok{)] <-}\StringTok{ "3"}

\NormalTok{event_data}\OperatorTok{$}\NormalTok{CROPDMGEXP[}\KeywordTok{grep}\NormalTok{(}\StringTok{"M"}\NormalTok{, }\DataTypeTok{x =}\NormalTok{ event_data}\OperatorTok{$}\NormalTok{CROPDMGEXP,}\DataTypeTok{ignore.case =} \OtherTok{TRUE}\NormalTok{)] <-}\StringTok{ "6"}

\NormalTok{event_data}\OperatorTok{$}\NormalTok{CROPDMGEXP[}\KeywordTok{grep}\NormalTok{(}\StringTok{"B"}\NormalTok{, }\DataTypeTok{x =}\NormalTok{ event_data}\OperatorTok{$}\NormalTok{CROPDMGEXP,}\DataTypeTok{ignore.case =} \OtherTok{TRUE}\NormalTok{)] <-}\StringTok{ "9"}

\CommentTok{#Convert unit field to numeric}
\NormalTok{event_data}\OperatorTok{$}\NormalTok{CROPDMGEXP <-}\StringTok{ }\KeywordTok{as.numeric}\NormalTok{(event_data}\OperatorTok{$}\NormalTok{CROPDMGEXP)}
\end{Highlighting}
\end{Shaded}

\begin{verbatim}
## Warning: NAs introduced by coercion
\end{verbatim}

\begin{Shaded}
\begin{Highlighting}[]
\CommentTok{#Creating a new variable Crop damage that would be the product of the dollar value and the unit}
\NormalTok{event_data}\OperatorTok{$}\NormalTok{Crop_Damage <-}\StringTok{ }\NormalTok{event_data}\OperatorTok{$}\NormalTok{CROPDMG }\OperatorTok{*}\StringTok{ }\DecValTok{10}\OperatorTok{^}\NormalTok{event_data}\OperatorTok{$}\NormalTok{CROPDMGEXP}
\end{Highlighting}
\end{Shaded}

\hypertarget{analysis}{%
\subsection{Analysis}\label{analysis}}

\hypertarget{aggregating-health-related-impacts-by-event-type}{%
\subsubsection{Aggregating health related impacts by event
type}\label{aggregating-health-related-impacts-by-event-type}}

\begin{Shaded}
\begin{Highlighting}[]
\CommentTok{# Loading the dplyr package}
\KeywordTok{library}\NormalTok{(dplyr)}
\end{Highlighting}
\end{Shaded}

\begin{verbatim}
## 
## Attaching package: 'dplyr'
\end{verbatim}

\begin{verbatim}
## The following objects are masked from 'package:stats':
## 
##     filter, lag
\end{verbatim}

\begin{verbatim}
## The following objects are masked from 'package:base':
## 
##     intersect, setdiff, setequal, union
\end{verbatim}

\begin{Shaded}
\begin{Highlighting}[]
\CommentTok{# Creating a new data frame for aggregated health values (Fatalities + Injuries)}

\NormalTok{event_type_health_total <-}\StringTok{ }\NormalTok{event_data }\OperatorTok\StringTok{ }\KeywordTok{group_by}\NormalTok{(Event) }\OperatorTok\StringTok{ }\KeywordTok{summarise}\NormalTok{(}\DataTypeTok{impact_on_public_health =} \KeywordTok{sum}\NormalTok{(FATALITIES, INJURIES, }\DataTypeTok{na.rm =} \OtherTok{TRUE}\NormalTok{))}


\CommentTok{# Creating another data frame for aggregated economic impact (property + crop damage) values by event type}

\NormalTok{event_type_economy <-}\StringTok{ }\NormalTok{event_data }\OperatorTok\StringTok{ }\KeywordTok{group_by}\NormalTok{(Event) }\OperatorTok\StringTok{ }\KeywordTok{summarise}\NormalTok{(}\DataTypeTok{impact_on_economy =} \KeywordTok{sum}\NormalTok{(Property_Damage, Crop_Damage, }\DataTypeTok{na.rm =} \OtherTok{TRUE}\NormalTok{))}
\end{Highlighting}
\end{Shaded}

\hypertarget{result}{%
\subsection{Result}\label{result}}

Finally, we can create histogram plots to show the severe weather
event(s) that contribute to the most adverse impact on health and
ecnomoy.

\begin{Shaded}
\begin{Highlighting}[]
\CommentTok{# Loading the ggplot2 package}
\KeywordTok{library}\NormalTok{(ggplot2)}

\CommentTok{# Plot for health impact}
\KeywordTok{ggplot}\NormalTok{(}\DataTypeTok{data =}\NormalTok{ event_type_health_total, }\KeywordTok{aes}\NormalTok{(}\DataTypeTok{y =}\NormalTok{ Event, }\DataTypeTok{x =}\NormalTok{ impact_on_public_health, }\DataTypeTok{fill =}\NormalTok{ Event)) }\OperatorTok{+}\StringTok{ }\KeywordTok{geom_bar}\NormalTok{(}\DataTypeTok{stat =} \StringTok{"identity"}\NormalTok{) }\OperatorTok{+}\StringTok{ }\KeywordTok{xlab}\NormalTok{(}\StringTok{"Total Impact on Public Health"}\NormalTok{) }\OperatorTok{+}\StringTok{ }\KeywordTok{ylab}\NormalTok{(}\StringTok{"Event Type"}\NormalTok{) }\OperatorTok{+}\StringTok{ }\KeywordTok{ggtitle}\NormalTok{(}\StringTok{"Impact of Severe Weather Events on Public Health"}\NormalTok{)}
\end{Highlighting}
\end{Shaded}

\includegraphics{Analysis-of-Severe-Weather-Events-on-Public-Health-and-Economy-for-Communities-and-Municipalities-in-the-United-States_files/figure-latex/histhealth-1.pdf}

\begin{Shaded}
\begin{Highlighting}[]
\CommentTok{# Plot for economic impact}
\KeywordTok{ggplot}\NormalTok{(}\DataTypeTok{data =}\NormalTok{ event_type_economy, }\KeywordTok{aes}\NormalTok{(}\DataTypeTok{y =}\NormalTok{ Event, }\DataTypeTok{x =}\NormalTok{ impact_on_economy, }\DataTypeTok{fill =}\NormalTok{ Event)) }\OperatorTok{+}\StringTok{ }\KeywordTok{geom_bar}\NormalTok{(}\DataTypeTok{stat =} \StringTok{"identity"}\NormalTok{) }\OperatorTok{+}\StringTok{ }\KeywordTok{xlab}\NormalTok{(}\StringTok{"Total Impact on Economy"}\NormalTok{) }\OperatorTok{+}\StringTok{ }\KeywordTok{ylab}\NormalTok{(}\StringTok{"Event Type"}\NormalTok{) }\OperatorTok{+}\StringTok{ }\KeywordTok{ggtitle}\NormalTok{(}\StringTok{"Impact of Severe Weather Events on Economy"}\NormalTok{)}
\end{Highlighting}
\end{Shaded}

\includegraphics{Analysis-of-Severe-Weather-Events-on-Public-Health-and-Economy-for-Communities-and-Municipalities-in-the-United-States_files/figure-latex/histhealth-2.pdf}

\hypertarget{summary}{%
\subsection{Summary}\label{summary}}

Based on the results of the bar plots, the highest severe weather events
that adversly impacts public health are \textbf{Tornados} and the
economy are \textbf{Floods}.

\end{document}
